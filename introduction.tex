\section{Introduction}
Dans le contexte de la transition numérique d’EDF et dans sa volonté d’exploiter l’ensemble de ses données, il est aujourd’hui nécessaire de tester des méthodes de fouille de texte. Une chaîne de traitement a été mise en place pour extraire et analyser des informations à partir de rapports de maintenance. Des tests ont permis de mettre en évidence l’apport de Word2Vec \cite{DBLP:journals/corr/abs-1301-3781}, pour l’aide à la constitution de ressources lexicales. L’automatisation du processus met en exergue des éléments auparavant noyés dans la masse des données. Le gain est double : une économie de temps est réalisée grâce à la proposition de termes candidats au peuplement de lexiques ; nous produisons actuellement une sortie RDF avec une ontologie associée et proposons une visualisation sous forme de graphe avec l’outil SemVue. Pour chaque terme du corpus, des candidats sont donnés après prétraitements. Le gain est également qualitatif : des synonymes, des abréviations, des possibles fautes d’orthographe et des phénomènes de multilinguisme sont retournés par le système. Nous développons actuellement CuriosiText, une application web pour l'aide au peuplement d'ontologie par des utilisateurs non experts. 
