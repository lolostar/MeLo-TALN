\section{Exploration de termes sous forme de graphe RDF}

Afin de visualiser les résultats de Word2Vec formatés en RDF, nous avons créé une première ontologie modélisée sur Word2Vec. Elle est un premier palier en vue du peuplement d'ontologies métiers.

\begin{figure}[tb]
    \begin{center}
        \includegraphics[width=6cm]{figures/gtype}
    \end{center}
    \caption{Graphe Type de l’ontologie pour Word2Vec}\label{fig:gtype}
\end{figure}


Nous souhaitions une visualisation sous forme de graphe où pour chaque mot du corpus, considéré comme étant un Représentant, nous affichons ses quinze Candidats Termes les plus similaires.

Pour visualiser ces résultats, nous avons utilisé l'outil SemVue développé en 2014 à EDF. Cet outil propose une interface de navigation dans un graphe RDF extrêmement facile à mettre en œuvre à chaque étape du développement d’un entrepôt de données. Le cœur de l’interface est défini à l’aide d’une ontologie associée à la modélisation métier proprement dite dans un entrepôt de données \cite{mnpho14parallel-materialisation-RDFox} et permet ainsi à l’interface web de lancer des requêtes SPARQL. La stratégie d’affichage de SemVue est définie à l’aide d’axiomes OWL, ceux-ci configurant dynamiquement le sous-graphe à exposer à l’utilisateur.

Cette première visualisation nous a permis d'optimiser le paramétrage de Word2Vec et de valider qualitativement nos résultats. Ainsi nous observons pour le terme changement \ref{fig:w2v}

\begin{figure}[tb]
    \begin{center}
        \includegraphics[width=6cm]{figures/w2v}
    \end{center}
    \caption{Graphe sur le mot changement}\label{fig:w2v}
\end{figure}


Afin de visualiser les résultats de Word2Vec formatés en RDF, nous avons créé une première ontologie modélisée sur Word2Vec. Elle est un premier palier en vue du peuplement d'ontologies métiers.
