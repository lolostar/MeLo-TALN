\section{Conclusion et perspectives}
Les premiers résultats de cette étude se sont révélés très prometteurs relativement à nos besoins métiers. Nous avons pu utiliser des méthodes de plongement de mots, de visualisation par graphes et de peuplement d'ontologie pour un cas industriel concret. Il nous reste à évaluer quantitivement ces résultats sur l'ensemble de nos données. Nous souhaiterions également évaluer l'apprentissage de Word2Vec avec et sans étiquetage morphosyntaxique pour en démontrer l'apport. Par ailleurs, il serait intéressant d'utiliser l'ontologie peuplée semi-automatiquement comme corpus de référence pour de l'apprentissage supervisé en vue de la détection automatique de termes pertinents. Cette approche réduirait la tâche d'annotation en classes et en relations.
Le paramétrage de Word2Vec sur les expressions multi-mots fait également partie de nos perspectives afin de gagner en pertinence.
