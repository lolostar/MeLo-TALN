\section{Contexte de l'étude}
Une chaîne de traitement de fouille de texte a été mise en place en 2017 à l’aide de la plateforme GATE (General Architecture for Text Engineering). Cette chaîne a pour objectif d’extraire automatiquement des actions de maintenance réalisées sur des composants (exemple : remplacement de l’anémomètre). Dans un premier temps et afin de démontrer la faisabilité de notre cas d’étude, les gazetteers (ou dictionnaires) de cette chaîne ont été développés manuellement. Des experts les ont constitués en pointant les termes pertinents d’une partie du corpus des comptes rendus de maintenance. En vue du passage à l’échelle, nous avons cherché à automatiser cette étape coûteuse en temps ; nous avons ainsi adopté la méthode Word2Vec pour extraire automatiquement les termes similaires sur l’ensemble du corpus et augmenter le rappel de la chaîne de traitement.
