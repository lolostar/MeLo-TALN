\section{Evaluation}
Nous avons réalisé une analyse qualitative en considérant les termes présents dans le lexique métier constitué manuellement. L'apport de Word2Vec a ainsi pu être démontré, certains termes ayant été omis par l'annotateur. Nous avons ainsi détecté automatiquement plusieurs éléments pertinents à savoir : des pluriels (ex : \texttt{moteur} / \texttt{moteurs}), des abréviations (ex : \texttt{transfo} / \texttt{transformateur}), des mots collés (\texttt{bloc} / \texttt{dublock}), des fautes de frappe (ex : \texttt{changement} / \texttt{hamgement}), des hyponymes (pour \texttt{équipements} on retrouve \texttt{modem}, \texttt{serveur}), des mots du même radical (ex : \texttt{ventilateur} / \texttt{ventilation}), des synonymes (ex : pour \texttt{remplacement} on retrouve \texttt{changement}, \texttt{échange}) et des phénomènes de multilinguisme (ex \texttt{contrôle} / \texttt{cheik} / \texttt{check}). 
Il nous reste à présent à effectuer une analyse quantitative pour établir le taux de rappel et de précision en injectant ces lexiques à notre chaîne de traitement. Cette étape pourra être facilitée par nos outils d’exploration de termes et de peuplement, respectivement SemVue et CuriosiText.
