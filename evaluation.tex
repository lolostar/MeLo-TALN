\section{Evaluation}
Nous avons réalisé une analyse qualitative en considérant les termes présents dans le lexique métier constitué manuellement. L'apport de Word2Vec a ainsi pu être démontré, certains termes ayant été omis par l'humain. Nous avons ainsi pu détecter automatiquement les types d’éléments suivants :

\begin{enumerate}
\item des pluriels (ex : « moteur », « moteurs »)
\item 	des abréviations (ex : pour « transfo », « transformateur », qui n’était pas présent dans le lexique constitué manuellement)
\item 	des fautes/mots collés (« bloc » et « dublock »)
\item 	des hyponymes (pour « équipements » on retrouve « modem », « serveur » etc.)
\item 	des mots de la même famille (ex : « ventilateur » et « ventilation »)
\item 	des synonymes (ex : pour « remplacement » on retrouve avec W2V « changement », « échange » etc.
\item 	des phénomènes de multilinguisme : pour le terme « contrôle » : « controle » sans accent, termes anglais associés plus ou moins bien orthographiés « cheik » et « check ».
\item 	pour le terme « arrêt » on retrouve « stop »

\end{enumerate}

Il nous reste à présent à effectuer une analyse quantitative poussée pour établir le taux de rappel et de précision en injectant ces lexiques à notre chaîne de traitement. Cette étape pourra être facilitée par notre outil d’exploration de termes (Partie 4) et par CuriosiText (Partie 5).
